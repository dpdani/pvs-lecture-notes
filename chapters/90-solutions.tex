\chapter{Solutions}

\section{The simple hashmap example}\label{sol:simple-hashmap}

\lstinputlisting{exercises/01 hashmap.pvs}

\section{Exercise \ref{ex:chapter1-1}}
\lstinputlisting{exercises/02chapter1exercises.pvs}

\section{Chapter \ref{ch:concurrent-reference-counting} Pseudocode}\label{sec:refcount-full-pseudocode}
\lstinputlisting[numbers=left, stepnumber=5, numberfirstline=true]{exercises/02 refcount.c}

\section{Exercise \ref{ex:gc-cycle-4}}
%Not much would change.
%Looking at the pseudocode, we can see that the concurrency-relevant parts had been already taken care of in other steps.
%That is, the ``locking'' and ``unlocking'' sequences are not involved in this line.
%Therefore, adding more details in this part would only serve to model the behavior of the runtime execution overly precisely.

\section{Chapter \ref{ch:concurrent-reference-counting} PVS Specification}\label{sec:refcount-full-pvs-specification}
\lstinputlisting{exercises/02 refcount.pvs}
